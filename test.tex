\documentclass[a4paper,12pt]{article}
\usepackage[utf8]{inputenc}
\usepackage[T1]{fontenc}
\usepackage{amsmath}
\usepackage[french]{babel}
\usepackage{xcolor}
\usepackage{geometry}
\geometry{top=2cm, bottom=2cm, left=2cm, right=2cm}
\usepackage{array}

\begin{document}

\noindent

\begin{center}
\hrule
\vspace{.1cm}
\begin{tabular}{>{\raggedright}p{0.45\textwidth} >{\raggedleft\arraybackslash}p{0.45\textwidth}}
\textbf{Université Gustave Eiffel} & \textbf{Licence L1} \\
\textbf{1er semestre 2023/2024} & \textbf{Calcul différentiel et intégral} \\
\end{tabular}

\vspace{.5cm}
\textbf{Contrôle Continu 1 du 09 Novembre 2023} 
\\
\textbf{Correction} \\
\textit{Les documents, calculatrices, portables et objets connectés sont interdits}
\vspace{.5cm}
\textcolor{gray}{\hrule}
\end{center}


\newcommand{\exercice}[2]{\vspace{1cm}
\noindent \large \textbf{Exercice #1} \textit{(#2 points)}
\vspace{0.1cm}
\hrule
\vspace{.5cm}
}


\exercice{1}{5}

\begin{enumerate}
    \item On considère l'équation $z^2+z+1-i = 0$ \\ On a $\Delta = (1)^2-4\cdot 1 \cdot (1-i) = -3+4i$ \\ On note $\delta$ une racine de $\Delta$ telle que $\delta = x+iy$ \\
    \(\delta^2 = \Delta \Rightarrow 
   \left\{
\begin{array}{l}
    x^2 - y^2 + 2ixy = -3 + 4i \\
    x^2 + y^2 = |\delta|^2 = |\Delta| = \sqrt{3^2 + 4^2} = 5
\end{array}
\right. \\
\Rightarrow \begin{cases}
    x^2 - y^2 = -3 \\
    2xy = 4 \\ 
    x^2 + y^2 = 5
\end{cases}
\Rightarrow
\begin{cases}
    x^2 - y^2 = -3 \\
    x^2 + y^2 = 5
\end{cases}
\Rightarrow 
\begin{cases}
2x^2 = 2 \\
2y^2 = 8
\end{cases}
\\
\Rightarrow \begin{cases}
x = \pm 1 \\
y = \pm 2

\end{cases}
   \) \\ \\
Ainsi, les racines de $\Delta$ sont $1+2i$ et $1-2i$. \\
Soient $z_1, z_2$ les solutions de l'équation:\\
$
z_1 = \frac{-1-(1+2i)}{2} = -1-i  \\
z_2 = \frac{-1+(1-2i)}{2} = i
$
\item On a:  \begin{itemize}
    \item Pour $z_1$, $Re(z_1) = 0$ et $Im(z_1) = 1$. On a
    $|z_1| = 1$ \\
    \begin{cases}
    \cos(\theta) = \frac{Re(z_1)}{|z_1|} = \frac{0}{1} \\
    \sin(\theta) = \frac{Im(z_1)}{|z_1|} = \frac{1}{1}
    \end{cases} \Rightarrow \begin{cases}
    \cos(\theta) = 0 \\
    \sin(\theta) = 1 
    \end{cases} \Rightarrow \theta = \frac{\pi}{2}$
    \\ $\Rightarrow$ Ainsi $z_1 = e^{i\frac{\pi}{2}}$
    \item Pour $z_2$, $Re(z_2) = -1$ et $Im(z_2) = -1$. On a $|z_2| = 2$ \\
    \begin{cases}
        \cos(\theta) = \frac{Re(z_2)}{|z_2|}=\frac{-1}{\sqrt{2}} = \frac{-\sqrt{2}}{2} \\
        \sin(\theta) = \frac{Im(z_2)}{|z_2|}=\frac{-1}{\sqrt{2}} = \frac{-\sqrt{2}}{2}
    \end{cases}$ \\
    $\Rightarrow$ Ainsi, $z_2 = \sqrt{2}e^{i\frac{\pi}{5}}$ 
    \clearpage
\end{itemize}
3. On cherche à résoudre $z^8+z^4+1-i=0$. On a: \\
$z^8+z^4+1-i=0 \Leftrightarrow \begin{cases} 
Z^2 + Z + 1 - i= 0 \\
Z = z^4
\end{cases}$
\\ D'après la question (1):  \\
\begin{cases}
Z \in \{i;-1-i\} \\
Z = z^4
\end{cases}
\end{enumerate}
\\ Ainsi, on doit résoudre:\\
\begin{matrix}
z^4 = i = e^{i\frac{\pi}{2}} & \text{et} & z^4 = -1-i = \sqrt{2}e^{i\frac{\pi}{3}}
\end{matrix} \\
\begin{cases}
z^4 = e^{i\frac{\pi}{2}} \\
z = \rho e^{i\theta}
\end{cases} \Leftrightarrow
\begin{cases}
    \rho^4 = 1 \\
    4\theta = \frac{\pi}{2}[2\pi]
\end{cases}
\Rightarrow \begin{cases} 
    \rho = 1 \\
    \theta = \frac{\pi}{8}+\frac{2k\pi}{4} / k \in \mathbb{Z}
\end{cases}
\\ z^4 = e^{i\frac{\pi}{2}}\Leftrightarrow z \in \{\frac{5\pi}{8};\frac{9\pi}{8};\frac{13\pi}{8}\}$
\\ et:
$\begin{cases}
    z^4 = \sqrt{}
\end{cases}$

\end{document}
